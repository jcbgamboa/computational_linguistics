\documentclass[a4paper,11pt]{scrartcl}

%\renewcommand{\arraystretch}{1.5}


%\lvsemester{SS 2021}
%\lvname{Computational Linguistics}
%\zeit{60 minutes}
%\datum{July 21\textsuperscript{th} 2021}

\usepackage{enumitem}
\usepackage[outputdir=out]{minted}
\usepackage{xcolor} % to access the named colour LightGray
\definecolor{LightGray}{gray}{0.9}


% Load the setspace package
\usepackage{setspace}
% Using \doublespacing in the preamble 
% changes text to double line spacing
\doublespacing


\title{W2 Assignment -- NLTK}
\subtitle{Computational Linguistics}

\author{John Gamboa}
\date{\today}

\setkomafont{author}{\sffamily}
\setkomafont{date}{\sffamily}



\setlength{\parindent}{0pt}



\begin{document}
\maketitle

% !TeX spellcheck = en_US
\section{NLTK basics}

\subsection{Text objects}

Consider the following piece of code:

{\singlespacing
\begin{minted}[bgcolor=LightGray,linenos,fontsize=\footnotesize]{python}
import nltk
import nltk.book

t = nltk.book.text7
print(t)
type(t)
len(t)

set(t)
len(set(t))

len(set(t)) / len(t)

t.concordance('walk')
t.index('walk')
\end{minted}
}

Answer \textit{TRUE} or \textit{FALSE}

\begin{enumerate}[label=\alph*)]
\singlespacing%\onehalfspacing
\item Line 5 will output the value
\begin{minted}[bgcolor=LightGray,fontsize=\footnotesize]{python}
<Text: Wall Street Journal>
\end{minted}
indicating that \verb|text7| was data collected from the Wall Street Journal.

\verb|answer|: \_\_\_\_\_\_\_\_\_\_\_\_\_\_\_\_\_\_\_\_\_\_\_\_\_\_\_

\item Line 6 will output the value
\begin{minted}[bgcolor=LightGray,fontsize=\footnotesize]{python}
nltk.text.Text
\end{minted}
indicating that \verb|t| is of the type \verb|Text|.

\verb|answer|: \_\_\_\_\_\_\_\_\_\_\_\_\_\_\_\_\_\_\_\_\_\_\_\_\_\_\_

\item Line 7 will output the value
\begin{minted}[bgcolor=LightGray,fontsize=\footnotesize]{python}
100676
\end{minted}
indicating that \verb|t| contains 100676 characters.

\verb|answer|: \_\_\_\_\_\_\_\_\_\_\_\_\_\_\_\_\_\_\_\_\_\_\_\_\_\_\_

\item Line 5 will output the value
\begin{minted}[bgcolor=LightGray,fontsize=\footnotesize]{python}
<Text: Moby Dick>
\end{minted}
indicating that \verb|text7| is the Moby Dick book.

\verb|answer|: \_\_\_\_\_\_\_\_\_\_\_\_\_\_\_\_\_\_\_\_\_\_\_\_\_\_\_

\item Line 6 will output the value
\begin{minted}[bgcolor=LightGray,fontsize=\footnotesize]{python}
list
\end{minted}
indicating that \verb|t| is of the type \verb|list|

\verb|answer|: \_\_\_\_\_\_\_\_\_\_\_\_\_\_\_\_\_\_\_\_\_\_\_\_\_\_\_

\item Line 9 will output a data structure of the type \verb|set|
containing all words in \verb|t|.

\verb|answer|: \_\_\_\_\_\_\_\_\_\_\_\_\_\_\_\_\_\_\_\_\_\_\_\_\_\_\_

\item Line 7 will output the value
\begin{minted}[bgcolor=LightGray,fontsize=\footnotesize]{python}
100676
\end{minted}
indicating that \verb|t| contains 100676 words.

\verb|answer|: \_\_\_\_\_\_\_\_\_\_\_\_\_\_\_\_\_\_\_\_\_\_\_\_\_\_\_

\item Line 9 will set \verb|t| as the default variable to be used
in all function calls to the NLTK.

\verb|answer|: \_\_\_\_\_\_\_\_\_\_\_\_\_\_\_\_\_\_\_\_\_\_\_\_\_\_\_

\item Line 6 will output the value
\begin{minted}[bgcolor=LightGray,fontsize=\footnotesize]{python}
set
\end{minted}
indicating that \verb|t| is of the type \verb|set|.

\verb|answer|: \_\_\_\_\_\_\_\_\_\_\_\_\_\_\_\_\_\_\_\_\_\_\_\_\_\_\_

\item Line 7 will output the value
\begin{minted}[bgcolor=LightGray,fontsize=\footnotesize]{python}
3
\end{minted}
because that \verb|t| contains the words ``Wall", ``Street" and
``Journal" words.

\verb|answer|: \_\_\_\_\_\_\_\_\_\_\_\_\_\_\_\_\_\_\_\_\_\_\_\_\_\_\_

\item Line 10 calculates the number of unique words in \verb|t|.

\verb|answer|: \_\_\_\_\_\_\_\_\_\_\_\_\_\_\_\_\_\_\_\_\_\_\_\_\_\_\_

\item Line 12 calculates the number of sections in the Wall Street Journal data.

\verb|answer|: \_\_\_\_\_\_\_\_\_\_\_\_\_\_\_\_\_\_\_\_\_\_\_\_\_\_\_

\item Line 10 calculates the number of sections in the Wall Street Journal data.

\verb|answer|: \_\_\_\_\_\_\_\_\_\_\_\_\_\_\_\_\_\_\_\_\_\_\_\_\_\_\_

\item Line 12 provides a measure of the lexical diversity of \verb|t|.

\verb|answer|: \_\_\_\_\_\_\_\_\_\_\_\_\_\_\_\_\_\_\_\_\_\_\_\_\_\_\_

\item Line 14 will output the list
\begin{minted}[bgcolor=LightGray,fontsize=\footnotesize]{python}
[walk, walks, walking, walked]
\end{minted}
indicating all the possible endings of the word ``walk".

\item Line 14 will output all the contexts in which the word ``walk" was used
in \verb|t|.

\verb|answer|: \_\_\_\_\_\_\_\_\_\_\_\_\_\_\_\_\_\_\_\_\_\_\_\_\_\_\_

\item Line 14 will output all the contexts in which the word ``walk" was used in
\verb|t|, along with whether it was used with the correct ending.

\verb|answer|: \_\_\_\_\_\_\_\_\_\_\_\_\_\_\_\_\_\_\_\_\_\_\_\_\_\_\_

\item The output of the line 14 follows the Key Word outside of Context format.

\verb|answer|: \_\_\_\_\_\_\_\_\_\_\_\_\_\_\_\_\_\_\_\_\_\_\_\_\_\_\_

\item The output of the line 14 follows the Key Word in Context format.

\verb|answer|: \_\_\_\_\_\_\_\_\_\_\_\_\_\_\_\_\_\_\_\_\_\_\_\_\_\_\_

\item Line 15 outputs a new data structure that allows for an efficient search
of all occurrences of the word ``walk".

\verb|answer|: \_\_\_\_\_\_\_\_\_\_\_\_\_\_\_\_\_\_\_\_\_\_\_\_\_\_\_

\item Line 15 outputs a list containing the index of all occurrences of the word
``walk".

\verb|answer|: \_\_\_\_\_\_\_\_\_\_\_\_\_\_\_\_\_\_\_\_\_\_\_\_\_\_\_

\item Line 15 outputs the index of the first occurrences of the word``walk".

\verb|answer|: \_\_\_\_\_\_\_\_\_\_\_\_\_\_\_\_\_\_\_\_\_\_\_\_\_\_\_

\end{enumerate}



\subsection{Plots 1}

Consider the following piece of code:

{\singlespacing
\begin{minted}[bgcolor=LightGray,linenos,fontsize=\footnotesize]{python}
import nltk
import nltk.book

t = nltk.book.text7
\end{minted}
}

Write the line of code that would create a plot that shows the positions of
occurrences in \verb|t| of the word ``knowledge".

\verb|command|: \_\_\_\_\_\_\_\_\_\_\_\_\_\_\_\_\_\_\_\_\_\_\_\_\_\_\_\_\_\_\_\_\_\_\_\_\_\_\_\_\_\_\_\_\_\_\_\_\_\_\_\_\_\_\_\_\_\_\_\_\_\_\_\_\_\_\_\_\_\_


\subsection{Plots 2}

Consider the following piece of code:

{\singlespacing
\begin{minted}[bgcolor=LightGray,linenos,fontsize=\footnotesize]{python}
import nltk
import nltk.book

t = nltk.book.text7
freqs = nltk.FreqDist(t)
\end{minted}
}

Write the line of code that would create a plot showing the frequencies of
the 20 most common words \verb|t|.

\verb|command|: \_\_\_\_\_\_\_\_\_\_\_\_\_\_\_\_\_\_\_\_\_\_\_\_\_\_\_\_\_\_\_\_\_\_\_\_\_\_\_\_\_\_\_\_\_\_\_\_\_\_\_\_\_\_\_\_\_\_\_\_\_\_\_\_\_\_\_\_\_\_


\end{document}

