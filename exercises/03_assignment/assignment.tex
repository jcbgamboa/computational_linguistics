\documentclass[a4paper,11pt]{scrartcl}

%\renewcommand{\arraystretch}{1.5}


%\lvsemester{SS 2021}
%\lvname{Computational Linguistics}
%\zeit{60 minutes}
%\datum{July 21\textsuperscript{th} 2021}

\usepackage{enumitem}
\usepackage[outputdir=out]{minted}
\usepackage{xcolor} % to access the named colour LightGray
\definecolor{LightGray}{gray}{0.9}


% Load the setspace package
\usepackage{setspace}
% Using \doublespacing in the preamble 
% changes text to double line spacing
\doublespacing


\title{W3 Assignment -- Corpus Linguistics 1}
\subtitle{Computational Linguistics}

\author{John Gamboa}
\date{\today}

\setkomafont{author}{\sffamily}
\setkomafont{date}{\sffamily}



\setlength{\parindent}{0pt}



\begin{document}
\maketitle

% !TeX spellcheck = en_US
\section{What is Corpus Linguistics?}

Answer \textit{TRUE} or \textit{FALSE}

\begin{enumerate}[label=\alph*)]
\singlespacing%\onehalfspacing
\item We defined Corpus Linguistics as

\textit{A methodology that analyses corpora to address linguistic questions}

This definition is an attempt to summarize the discussion from Gries (2009)

\verb|answer|: \_\_\_\_\_\_\_\_\_\_\_\_\_\_\_\_\_\_\_\_\_\_\_\_\_\_\_


\item Most researchers believe that Corpus Linguistics is not just a new methodoly, but a completely new research enterprise, or even a new philosophical approach to Linguistics

\verb|answer|: \_\_\_\_\_\_\_\_\_\_\_\_\_\_\_\_\_\_\_\_\_\_\_\_\_\_\_


\item There is not much debate about what Corpus Linguistics is

\verb|answer|: \_\_\_\_\_\_\_\_\_\_\_\_\_\_\_\_\_\_\_\_\_\_\_\_\_\_\_

\end{enumerate}


\section{Corpora}

\subsection{Additional Corpus Data}

Choose the correct alternative:

\begin{enumerate}[label=\alph*)]
\onehalfspacing
\item Corpora can contain additional information, such as Metadata and
      Annotations.

\item Metadata may contain, among others, the source of the data, or copyright
      information related to the corpus files

\item Metadata may contain, among others, the constituent tree of the sentences
      in the corpus files

\item Annotations may contain, among others, the age of each of the participants
      in the conversations recorded in the corpus

\item Both the first and second alternatives are correct

\item Both the first and the third alternatives are correct

\item Both the first and the fourth alternatives are correct

\item The first, the second, and the third alternatives are correct
\end{enumerate}

\verb|answer|: \_\_\_\_\_\_\_\_\_\_\_\_\_\_\_\_\_\_\_\_\_\_\_\_\_\_\_\_\_\_\_\_


\subsection{Criteria for defining a corpus}

As we saw, a corpus is a collection of texts. This collection needs to be:

\begin{enumerate}[label=\alph*)]
\onehalfspacing
\item \_\_\_\_\_\_\_\_\_\_\_\_\_\_\_\_\_\_\_\_\_\_\_\_\_\_\_\_\_\_\_\_ , i.e.,
      different parts of the linguistic variety are all present in the corpus
\item \_\_\_\_\_\_\_\_\_\_\_\_\_\_\_\_\_\_\_\_\_\_\_\_\_\_\_\_\_\_\_\_ , i.e.,
      these parts need to appear in the corpus according to the proportions
      they appear in real life
\item \_\_\_\_\_\_\_\_\_\_\_\_\_\_\_\_\_\_\_\_\_\_\_\_\_\_\_\_\_\_\_\_ , i.e.,
      ``the texts were spoken or written for some authentic communicative purpose"
\end{enumerate}



\subsection{Types of Corpora}

We saw 5 dimensions in which corpora can be classified. These are:

\begin{enumerate}[label=\alph*)]
\onehalfspacing
\item Depending on whether the corpus captures how the language varies over
      time or just keeps a snapshot of the language at a particular point of
      time.

\_\_\_\_\_\_\_\_\_\_\_\_\_\_\_\_\_\_\_\_\_\_\_\_\_\_\_\_\_\_\_\_ vs.
\_\_\_\_\_\_\_\_\_\_\_\_\_\_\_\_\_\_\_\_\_\_\_\_\_\_\_\_\_\_\_\_

\item Depending on whether the corpus focuses only in one language or has the
      same data in several languages

\_\_\_\_\_\_\_\_\_\_\_\_\_\_\_\_\_\_\_\_\_\_\_\_\_\_\_\_\_\_\_\_ vs.
\_\_\_\_\_\_\_\_\_\_\_\_\_\_\_\_\_\_\_\_\_\_\_\_\_\_\_\_\_\_\_\_

\item Depending on whether the corpus focus on the language as whole or only
      in a particular variety/dialect/register of the language.

\_\_\_\_\_\_\_\_\_\_\_\_\_\_\_\_\_\_\_\_\_\_\_\_\_\_\_\_\_\_\_\_ vs.
\_\_\_\_\_\_\_\_\_\_\_\_\_\_\_\_\_\_\_\_\_\_\_\_\_\_\_\_\_\_\_\_

\item Depending on whether or not the corpus has additional corpus data that
      represents a particular linguistic analysis

\_\_\_\_\_\_\_\_\_\_\_\_\_\_\_\_\_\_\_\_\_\_\_\_\_\_\_\_\_\_\_\_ vs.
\_\_\_\_\_\_\_\_\_\_\_\_\_\_\_\_\_\_\_\_\_\_\_\_\_\_\_\_\_\_\_\_

\item Depending on whether or not the corpus can be extended over time with
      new data

\_\_\_\_\_\_\_\_\_\_\_\_\_\_\_\_\_\_\_\_\_\_\_\_\_\_\_\_\_\_\_\_ vs.
\_\_\_\_\_\_\_\_\_\_\_\_\_\_\_\_\_\_\_\_\_\_\_\_\_\_\_\_\_\_\_\_
\end{enumerate}



\section{Answering Linguistic Questions}

\subsection{What do we normally calculate?}

What is normally calculated in Corpus Linguistics are 
\_\_\_\_\_\_\_\_\_\_\_\_\_\_\_\_\_\_\_\_\_\_\_\_\_\_\_\_\_\_\_\_.

Generally, we are interested in two types of them: 


\_\_\_\_\_\_\_\_\_\_\_\_\_\_\_\_\_\_\_\_\_\_\_\_\_\_\_\_\_\_\_\_ and
\_\_\_\_\_\_\_\_\_\_\_\_\_\_\_\_\_\_\_\_\_\_\_\_\_\_\_\_\_\_\_\_,

i.e., how often certain linguistic elements appear in the data, and how
often they appear along with other linguistic elements.


\section{NLTK / Python programming}

\subsection{Accessing Corpus Data}

In the class, we have seen that the NLTK has a subpackage called
\verb|gutenberg|, which you can use to get access to some books from the
Project Gutenberg. Consider the following lines of code (In), and their
output (Out):

{\singlespacing
\begin{minted}[bgcolor=LightGray,linenos,fontsize=\footnotesize]{python}
In [1]: import nltk
In [2]: nltk.corpus.gutenberg.fileids()
Out[2]: ['austen-emma.txt',
         'austen-persuasion.txt',
         'austen-sense.txt',
         'bible-kjv.txt','blake-poems.txt',
         'bryant-stories.txt',
         'burgess-busterbrown.txt',
         'carroll-alice.txt',
         'chesterton-ball.txt',
         'chesterton-brown.txt',
         'chesterton-thursday.txt',
         'edgeworth-parents.txt',
         'melville-moby_dick.txt',
         'milton-paradise.txt',
         'shakespeare-caesar.txt',
         'shakespeare-hamlet.txt',
         'shakespeare-macbeth.txt',
         'whitman-leaves.txt']
\end{minted}
}


Write the line of code you'd use to get all the words in the file \verb|'chesterton-thursday.txt'|:

\verb|code|: \_\_\_\_\_\_\_\_\_\_\_\_\_\_\_\_\_\_\_\_\_\_\_\_\_\_\_\_\_\_\_\_\_\_\_\_\_\_\_\_\_\_\_\_\_\_\_\_\_\_\_\_\_\_\_\_\_\_\_\_\_\_\_\_\_\_\_\_\_\_
  

Write the line of code you'd use to get all the sentences in the same file:

\verb|code|: \_\_\_\_\_\_\_\_\_\_\_\_\_\_\_\_\_\_\_\_\_\_\_\_\_\_\_\_\_\_\_\_\_\_\_\_\_\_\_\_\_\_\_\_\_\_\_\_\_\_\_\_\_\_\_\_\_\_\_\_\_\_\_\_\_\_\_\_\_\_

Write the line of code you'd use to get a string containing the entire file data, unsegmented by any NLTK algorithm:

\verb|code|: \_\_\_\_\_\_\_\_\_\_\_\_\_\_\_\_\_\_\_\_\_\_\_\_\_\_\_\_\_\_\_\_\_\_\_\_\_\_\_\_\_\_\_\_\_\_\_\_\_\_\_\_\_\_\_\_\_\_\_\_\_\_\_\_\_\_\_\_\_\_



\subsection{List comprehensions}

Assume the following lines of code have been run:

{\singlespacing
\begin{minted}[bgcolor=LightGray,linenos,fontsize=\footnotesize]{python}
num_list = [0,1,2,3,4,5,6,7,8,9]
str_list = ['a', 'b', 'c', 'd', 'e', 'f']
lst_list = [[1,2], [3,4], [5,6,7], [8], [9]]
\end{minted}
}


Decide whether the statements below are correct.
(Ideally, try out these lines in Jupyter notebook, to see what they do)

\begin{itemize}
\singlespacing
\item The line
\begin{minted}[bgcolor=LightGray,fontsize=\footnotesize]{python}
a = [2*i for i in num_list if i < 5]
\end{minted}
will produce the same effect as
\begin{minted}[bgcolor=LightGray,fontsize=\footnotesize]{python}
a = []
for i in num_list:
    if (i < 5):
        a.append(2*i)
\end{minted}

\verb|Is correct|: \_\_\_\_\_\_\_\_\_\_\_\_\_\_\_\_\_\_\_\_\_\_\_\_\_\_\_


\item The line
\begin{minted}[bgcolor=LightGray,fontsize=\footnotesize]{python}
a = [2*i for i in num_list]
\end{minted}
will produce the same effect as
\begin{minted}[bgcolor=LightGray,fontsize=\footnotesize]{python}
a = []
for 2*i in num_list:
    a.append(i)
\end{minted}

\verb|Is correct|: \_\_\_\_\_\_\_\_\_\_\_\_\_\_\_\_\_\_\_\_\_\_\_\_\_\_\_


\item The line
\begin{minted}[bgcolor=LightGray,fontsize=\footnotesize]{python}
a = [2*i for i in num_list]
\end{minted}
will produce the same effect as
\begin{minted}[bgcolor=LightGray,fontsize=\footnotesize]{python}
a = []
for i in 2*num_list:
    a.append(i)
\end{minted}

\verb|Is correct|: \_\_\_\_\_\_\_\_\_\_\_\_\_\_\_\_\_\_\_\_\_\_\_\_\_\_\_


\item The line
\begin{minted}[bgcolor=LightGray,fontsize=\footnotesize]{python}
a = [len(i) for i in lst_list]
\end{minted}
will produce the list
\begin{minted}[bgcolor=LightGray,fontsize=\footnotesize]{python}
[2,2,3,1,1]
\end{minted}
which is a list containing the length of each list inside \verb|lst_list|.

\verb|Is correct|: \_\_\_\_\_\_\_\_\_\_\_\_\_\_\_\_\_\_\_\_\_\_\_\_\_\_\_


\item The line
\begin{minted}[bgcolor=LightGray,fontsize=\footnotesize]{python}
a = [i for i in num_list + str_list]
\end{minted}
will produce the same as
\begin{minted}[bgcolor=LightGray,fontsize=\footnotesize]{python}
a = num_list + str_list
\end{minted}

\verb|Is correct|: \_\_\_\_\_\_\_\_\_\_\_\_\_\_\_\_\_\_\_\_\_\_\_\_\_\_\_


\item The line
\begin{minted}[bgcolor=LightGray,fontsize=\footnotesize]{python}
a = [2*i for i in num_list]
\end{minted}
will produce the same effect as
\begin{minted}[bgcolor=LightGray,fontsize=\footnotesize]{python}
a = []
for i in num_list:
    a.append(2*i)
\end{minted}

\verb|Is correct|: \_\_\_\_\_\_\_\_\_\_\_\_\_\_\_\_\_\_\_\_\_\_\_\_\_\_\_
\end{itemize}

\end{document}

