% !TeX spellcheck = en_US
\documentclass[a4paper,11pt]{scrartcl}

%\renewcommand{\arraystretch}{1.5}


%\lvsemester{SS 2021}
%\lvname{Computational Linguistics}
%\zeit{60 minutes}
%\datum{July 21\textsuperscript{th} 2021}

\usepackage{enumitem}
\usepackage[outputdir=out]{minted}
\usepackage{xcolor} % to access the named colour LightGray
\definecolor{LightGray}{gray}{0.9}


% Load the setspace package
\usepackage{setspace}
% Using \doublespacing in the preamble 
% changes text to double line spacing
\doublespacing


\title{W5 Assignment -- Tokenization}
\subtitle{Computational Linguistics}

\author{John Gamboa}
\date{\today}

\setkomafont{author}{\sffamily}
\setkomafont{date}{\sffamily}



\setlength{\parindent}{0pt}



\begin{document}
\maketitle

\section{Tokens and Types}

\subsection{Definitions}

Select the correct option:

\begin{enumerate}[label=\alph*)]
\singlespacing
\item The sentence
\begin{minted}[bgcolor=LightGray,fontsize=\footnotesize]{text}
my cellphone and your cellphone
\end{minted}
contains 4 tokens and 5 types

\item The sentence
\begin{minted}[bgcolor=LightGray,fontsize=\footnotesize]{text}
my cellphone and your cellphone
\end{minted}
contains 5 tokens and 5 types

\item The sentence
\begin{minted}[bgcolor=LightGray,fontsize=\footnotesize]{text}
my cellphone and your cellphone
\end{minted}
contains 5 tokens but only 4 types

\item The sentence
\begin{minted}[bgcolor=LightGray,fontsize=\footnotesize]{text}
my cellphone and your cellphone
\end{minted}
contains 5 tokens but only 3 types

\end{enumerate}

\verb|answer|: \_\_\_\_\_\_\_\_\_\_\_\_\_\_\_\_\_\_\_\_\_\_\_\_\_\_\_\_\_\_\_\_


\subsection{True/False}

Answer \textit{TRUE} or \textit{FALSE} for the following assertions:


\begin{itemize}
\singlespacing
\item The text
\begin{minted}[bgcolor=LightGray,fontsize=\footnotesize]{text}
my Cellphone and your cellphone
\end{minted}
contains 5 tokens and, debatably, either 4 or 5 types

\verb|answer|: \_\_\_\_\_\_\_\_\_\_\_\_\_\_\_\_\_\_\_\_\_\_\_\_\_\_\_\_\_\_\_\_

\item The text
\begin{minted}[bgcolor=LightGray,fontsize=\footnotesize]{text}
my, your, and their cellphone
\end{minted}
contains 7 tokens and 6 types

\verb|answer|: \_\_\_\_\_\_\_\_\_\_\_\_\_\_\_\_\_\_\_\_\_\_\_\_\_\_\_\_\_\_\_\_

\item The text
\begin{minted}[bgcolor=LightGray,fontsize=\footnotesize]{text}
Rosa's cellphone is blue
\end{minted}
contains, debatably,
either 4 tokens and 4 types
or 5 tokens and 5 types

\verb|answer|: \_\_\_\_\_\_\_\_\_\_\_\_\_\_\_\_\_\_\_\_\_\_\_\_\_\_\_\_\_\_\_\_
\end{itemize}






\section{Text manipulation / Regular Expressions}

\subsection{Instructions}

Consider the following lines, which are run before the lines in the
``Questions" section below.

\begin{minted}[bgcolor=LightGray,fontsize=\footnotesize]{python}
import re
string = "The man and the woman...the the... the dog and the cat"
\end{minted}

In the gap texts below, write the output of the code immediately preceding it.
For example, for the code

\begin{minted}[bgcolor=LightGray,fontsize=\footnotesize]{python}
print('example')
\end{minted}

you should write

\begin{minted}[bgcolor=LightGray,fontsize=\footnotesize]{python}
example
\end{minted}

(i.e., without the quotes that denote a string in Python)

Ideally, you should run those lines and try to see why you get each of
the outputs. If you have difficulties in understanding the meaning of the
Regular Expressions, remember to visit those websites I suggested in the
videos. 


\subsection{Questions}


\begin{minted}[bgcolor=LightGray,fontsize=\footnotesize]{python}
if(re.search('man', string):
    print('contains')
\end{minted}

\verb|answer|: \_\_\_\_\_\_\_\_\_\_\_\_\_\_\_\_\_\_\_\_\_\_\_\_\_\_\_\_\_\_\_\_
 

\begin{minted}[bgcolor=LightGray,fontsize=\footnotesize]{python}
re.sub('man', 'men', string)
\end{minted}

\verb|answer|: \_\_\_\_\_\_\_\_\_\_\_\_\_\_\_\_\_\_\_\_\_\_\_\_\_\_\_\_\_\_\_\_
 

\begin{minted}[bgcolor=LightGray,fontsize=\footnotesize]{python}
re.sub('the', 'a', string)
\end{minted}
  
\verb|answer|: \_\_\_\_\_\_\_\_\_\_\_\_\_\_\_\_\_\_\_\_\_\_\_\_\_\_\_\_\_\_\_\_
 

\begin{minted}[bgcolor=LightGray,fontsize=\footnotesize]{python}
re.sub('[a-z]', 'a', string)
\end{minted}
  
\verb|answer|: \_\_\_\_\_\_\_\_\_\_\_\_\_\_\_\_\_\_\_\_\_\_\_\_\_\_\_\_\_\_\_\_
 

\begin{minted}[bgcolor=LightGray,fontsize=\footnotesize]{python}
re.sub('[^a-z]', '_', string)
\end{minted}
  
\verb|answer|: \_\_\_\_\_\_\_\_\_\_\_\_\_\_\_\_\_\_\_\_\_\_\_\_\_\_\_\_\_\_\_\_
 

\begin{minted}[bgcolor=LightGray,fontsize=\footnotesize]{python}
re.sub('\W+', '_', string)
\end{minted}
  
\verb|answer|: \_\_\_\_\_\_\_\_\_\_\_\_\_\_\_\_\_\_\_\_\_\_\_\_\_\_\_\_\_\_\_\_
 

\begin{minted}[bgcolor=LightGray,fontsize=\footnotesize]{python}
re.sub('(the)', '_\\1_', string)
\end{minted}
  
\verb|answer|: \_\_\_\_\_\_\_\_\_\_\_\_\_\_\_\_\_\_\_\_\_\_\_\_\_\_\_\_\_\_\_\_
 

\begin{minted}[bgcolor=LightGray,fontsize=\footnotesize]{python}
re.sub('(the|man)', '_\\1_', string)
\end{minted}

\verb|answer|: \_\_\_\_\_\_\_\_\_\_\_\_\_\_\_\_\_\_\_\_\_\_\_\_\_\_\_\_\_\_\_\_


\end{document}

