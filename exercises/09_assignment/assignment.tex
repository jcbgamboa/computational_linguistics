% !TeX spellcheck = en_US
\documentclass[a4paper,11pt]{scrartcl}

%\renewcommand{\arraystretch}{1.5}


%\lvsemester{SS 2021}
%\lvname{Computational Linguistics}
%\zeit{60 minutes}
%\datum{July 21\textsuperscript{th} 2021}

\usepackage{graphicx}
\usepackage{enumitem}
\usepackage[outputdir=out]{minted}
\usepackage[dvipsnames]{xcolor} % to access the named colour LightGray
\definecolor{LightGray}{gray}{0.9}

\usepackage[colorlinks]{hyperref}
\hypersetup{
    linkbordercolor = {White},
    urlbordercolor = .,
    linkcolor = .,
    urlcolor = {Blue}
}




% Load the setspace package
\usepackage{setspace}
% Using \doublespacing in the preamble 
% changes text to double line spacing
\doublespacing


\title{W9 Assignment -- Distributional Semantics}
\subtitle{Computational Linguistics}

\author{John Gamboa}
\date{\today}

\setkomafont{author}{\sffamily}
\setkomafont{date}{\sffamily}



\setlength{\parindent}{0pt}


%% Adapted from
%% https://tex.stackexchange.com/questions/179197/framed-or-colored-box-with-text-and-margin-notes
%\usepackage[many]{tcolorbox}
%\newtcolorbox{story}[1][]{
%  top=-10pt,
%  width=\textwidth,
%  fonttitle=\bfseries,
%  breakable,
%  %boxrule=10pt,
%  %extrude right by=4cm,
%  fonttitle=\bfseries\color{Brown},
%  colframe=LightGray,
%  colback=LightGray!10,
%  #1}



\begin{document}
\maketitle


\section{Concepts}

\subsection{Homonymy and Polysemy}

Regarding the concepts of homonymy and polysemy, we could say that...

(choose \textit{TRUE} or \textit{FALSE})

\begin{enumerate}[label=\alph*)]
\singlespacing%\onehalfspacing

\item Polysemic words are words with a single meaning

\verb|answer|: \_\_\_\_\_\_\_\_\_\_\_\_\_\_\_\_\_\_\_\_\_\_\_\_\_\_\_\_\_\_\_\_

\item Homonymous words have meanings that are related

\verb|answer|: \_\_\_\_\_\_\_\_\_\_\_\_\_\_\_\_\_\_\_\_\_\_\_\_\_\_\_\_\_\_\_\_

\item If two words are homonymous, then they are written/pronounced the same,
      but have unrelated meanings

\verb|answer|: \_\_\_\_\_\_\_\_\_\_\_\_\_\_\_\_\_\_\_\_\_\_\_\_\_\_\_\_\_\_\_\_

\item Homonymous words are synonyms (i.e., homonymous words are words with the
      same meaning)

\verb|answer|: \_\_\_\_\_\_\_\_\_\_\_\_\_\_\_\_\_\_\_\_\_\_\_\_\_\_\_\_\_\_\_\_

\item Polysemic words are synonyms (i.e., homonymous words are words with the
      same meaning)

\verb|answer|: \_\_\_\_\_\_\_\_\_\_\_\_\_\_\_\_\_\_\_\_\_\_\_\_\_\_\_\_\_\_\_\_

\item Homonymous words are words with opposite meanings

\verb|answer|: \_\_\_\_\_\_\_\_\_\_\_\_\_\_\_\_\_\_\_\_\_\_\_\_\_\_\_\_\_\_\_\_

\item Polysemic words are words with opposite meanings

\verb|answer|: \_\_\_\_\_\_\_\_\_\_\_\_\_\_\_\_\_\_\_\_\_\_\_\_\_\_\_\_\_\_\_\_

\item Homonymy and polysemy are just the same thing

\verb|answer|: \_\_\_\_\_\_\_\_\_\_\_\_\_\_\_\_\_\_\_\_\_\_\_\_\_\_\_\_\_\_\_\_

\item A polysemic word is a word with multiple meanings that are related
      (a.k.a. "multiple senses")

\verb|answer|: \_\_\_\_\_\_\_\_\_\_\_\_\_\_\_\_\_\_\_\_\_\_\_\_\_\_\_\_\_\_\_\_
\end{enumerate}


\subsection{Examples}


Consider the following sentences

\begin{enumerate}[label=(\arabic*)]
\singlespacing%\onehalfspacing
\item \label{ex_good}  He was a kind man
\item \label{ex_type}  \href{https://knowyourmeme.com/memes/what-kind-of-sorcery-is-this}{What kind of sorcery is this?}
\item \label{ex_type2} \href{https://en.wiktionary.org/wiki/kind}{I'll pay in kind for his insult.}
\end{enumerate}

The usages of kind in the sentences \ref{ex_good} and \ref{ex_type} are an example of
\_\_\_\_\_\_\_\_\_\_\_\_\_\_\_\_\_\_\_\_\_\_\_\_\_\_\_\_\_\_\_\_

The usages o kind in the sentences \ref{ex_type} and \ref{ex_type2} are an example of
\_\_\_\_\_\_\_\_\_\_\_\_\_\_\_\_\_\_\_\_\_\_\_\_\_\_\_\_\_\_\_\_



\section{Co-occurrence matrix and its uses}

\subsection{Co-occurrence matrix}

Consider the following sentences: (which I am randomly writing, without any
thought)

{\singlespacing
\begin{minted}[bgcolor=LightGray,fontsize=\footnotesize]{text}
The quick brown fox and the lazy dog were never the same.
The brown fox jumped over the lazy dog several times over
and over again. But the lazy dog ignored it forever.
\end{minted}
}

Using a vocabulary of 5 words, construct a co-occurrence matrix
and input the vectors associated with each of the following words:


\verb|dog|:    \_\_\_\_\_\_\_\_\_\_\_\_\_\_\_\_\_\_\_\_\_\_\_\_\_\_\_\_\_\_\_\_

\verb|jumped|: \_\_\_\_\_\_\_\_\_\_\_\_\_\_\_\_\_\_\_\_\_\_\_\_\_\_\_\_\_\_\_\_

\verb|lazy|:   \_\_\_\_\_\_\_\_\_\_\_\_\_\_\_\_\_\_\_\_\_\_\_\_\_\_\_\_\_\_\_\_


\subsection{Cosine Similarity}

Let's assume that the following vectors were the correct answer for the previous questions

\verb|dog:    [5,2,3,4,1]|

\verb|jumped: [3,3,0,1,0]|

\verb|lazy:   [5,2,3,4,2]|

 

Calculate the cosine similarity between the following pairs of words:

$cos($\verb|dog|   ,\verb|jumped|$)$: \_\_\_\_\_\_\_\_\_\_\_\_\_\_\_\_\_\_\_\_\_\_\_\_\_\_\_\_\_\_\_\_

$cos($\verb|dog|   ,\verb|lazy|  $)$: \_\_\_\_\_\_\_\_\_\_\_\_\_\_\_\_\_\_\_\_\_\_\_\_\_\_\_\_\_\_\_\_

$cos($\verb|jumped|,\verb|lazy|  $)$: \_\_\_\_\_\_\_\_\_\_\_\_\_\_\_\_\_\_\_\_\_\_\_\_\_\_\_\_\_\_\_\_

\end{document}

