% !TeX spellcheck = en_US
\documentclass[a4paper,11pt]{scrartcl}

%\renewcommand{\arraystretch}{1.5}


%\lvsemester{SS 2021}
%\lvname{Computational Linguistics}
%\zeit{60 minutes}
%\datum{July 21\textsuperscript{th} 2021}

\usepackage{graphicx}
\usepackage{enumitem}
\usepackage[outputdir=out]{minted}
\usepackage[dvipsnames]{xcolor} % to access the named colour LightGray
\definecolor{LightGray}{gray}{0.9}

\usepackage[colorlinks]{hyperref}
\hypersetup{
    linkbordercolor = {White},
    urlbordercolor = .,
    linkcolor = .,
    urlcolor = {Blue}
}




% Load the setspace package
\usepackage{setspace}
% Using \doublespacing in the preamble 
% changes text to double line spacing
\doublespacing


\title{W10 Assignment -- Distributional Semantics II}
\subtitle{Computational Linguistics}

\author{John Gamboa}
\date{\today}

\setkomafont{author}{\sffamily}
\setkomafont{date}{\sffamily}



\setlength{\parindent}{0pt}


%% Adapted from
%% https://tex.stackexchange.com/questions/179197/framed-or-colored-box-with-text-and-margin-notes
%\usepackage[many]{tcolorbox}
%\newtcolorbox{story}[1][]{
%  top=-10pt,
%  width=\textwidth,
%  fonttitle=\bfseries,
%  breakable,
%  %boxrule=10pt,
%  %extrude right by=4cm,
%  fonttitle=\bfseries\color{Brown},
%  colframe=LightGray,
%  colback=LightGray!10,
%  #1}



\begin{document}
\maketitle


\section{Concepts}

\subsection{Lemmas e Lexemes}

Regarding the concepts of homonymy and polysemy, we could say that...

(choose \textit{TRUE} or \textit{FALSE})

\begin{enumerate}[label=\alph*)]
\singlespacing%\onehalfspacing

\item The set \{cat, cats\} forms the \textit{lemma} associated with the
      \textit{lexeme} cat

\verb|answer|: \_\_\_\_\_\_\_\_\_\_\_\_\_\_\_\_\_\_\_\_\_\_\_\_\_\_\_\_\_\_\_\_

\item The set \{cat, cats\} forms the \textit{lexeme} associated with the
      \textit{lemma} cat

\verb|answer|: \_\_\_\_\_\_\_\_\_\_\_\_\_\_\_\_\_\_\_\_\_\_\_\_\_\_\_\_\_\_\_\_

\item The set \{snow, snowed, snowing, snowed\} is a lexeme because it contains
      the forms associated with the word \textit{snow}

\verb|answer|: \_\_\_\_\_\_\_\_\_\_\_\_\_\_\_\_\_\_\_\_\_\_\_\_\_\_\_\_\_\_\_\_

\item Each of the elements of a given lexeme is a lemma

\verb|answer|: \_\_\_\_\_\_\_\_\_\_\_\_\_\_\_\_\_\_\_\_\_\_\_\_\_\_\_\_\_\_\_\_

\item Each of the elements of a given lexeme \textit{could, by convention} be a lemma

\verb|answer|: \_\_\_\_\_\_\_\_\_\_\_\_\_\_\_\_\_\_\_\_\_\_\_\_\_\_\_\_\_\_\_\_

\item Each of the elements of a given lemma is a lexeme

\verb|answer|: \_\_\_\_\_\_\_\_\_\_\_\_\_\_\_\_\_\_\_\_\_\_\_\_\_\_\_\_\_\_\_\_

\end{enumerate}


\section{Wordnet}

\subsection{Concepts}

Regarding the concepts of homonymy and polysemy, we could say that...

(choose \textit{TRUE} or \textit{FALSE})


\begin{enumerate}[label=\alph*)]
\singlespacing%\onehalfspacing

\item The Wordnet contains only words of the classes
      \textit{nouns}, \textit{verbs}, \textit{adjectives}, \textit{adverbs}

\verb|answer|: \_\_\_\_\_\_\_\_\_\_\_\_\_\_\_\_\_\_\_\_\_\_\_\_\_\_\_\_\_\_\_\_

\item The remaining classes of words (pronouns, prepositions, determiners) are
      referred to as \textit{content words}

\verb|answer|: \_\_\_\_\_\_\_\_\_\_\_\_\_\_\_\_\_\_\_\_\_\_\_\_\_\_\_\_\_\_\_\_

\item The Wordnet is organized into synsets in order to be able to make
      distinctions between homonymy and polysemy

\verb|answer|: \_\_\_\_\_\_\_\_\_\_\_\_\_\_\_\_\_\_\_\_\_\_\_\_\_\_\_\_\_\_\_\_

\item The Wordnet is organized into synsets

\verb|answer|: \_\_\_\_\_\_\_\_\_\_\_\_\_\_\_\_\_\_\_\_\_\_\_\_\_\_\_\_\_\_\_\_

\item The Wordnet makes no distinction between homonymy and polysemy

\verb|answer|: \_\_\_\_\_\_\_\_\_\_\_\_\_\_\_\_\_\_\_\_\_\_\_\_\_\_\_\_\_\_\_\_

\end{enumerate}


\subsection{Semantic Relations}

First, you will need the following code 

{\singlespacing
\begin{minted}[bgcolor=LightGray,fontsize=\footnotesize]{python}
from nltk.corpus import wordnet as wn
\end{minted}
}

Using the Wordnet, get

\begin{enumerate}[label=\alph*)]
\singlespacing%\onehalfspacing

\item The definition of the synset \textit{sun.n.01}

\verb|answer|: \_\_\_\_\_\_\_\_\_\_\_\_\_\_\_\_\_\_\_\_\_\_\_\_\_\_\_\_\_\_\_\_

\item An example for the synset \textit{get.v.01}

\verb|answer|: \_\_\_\_\_\_\_\_\_\_\_\_\_\_\_\_\_\_\_\_\_\_\_\_\_\_\_\_\_\_\_\_

\item All lemmas of the synset \textit{break.v.02}

\verb|answer|: \_\_\_\_\_\_\_\_\_\_\_\_\_\_\_\_\_\_\_\_\_\_\_\_\_\_\_\_\_\_\_\_

\item The relation between \textit{cold.n.01} and \textit{respiratory\_disease.n.01}

\verb|answer|: \_\_\_\_\_\_\_\_\_\_\_\_\_\_\_\_\_\_\_\_\_\_\_\_\_\_\_\_\_\_\_\_

\item The relation between \textit{freeze.v.02} and \textit{solidify.v.02}

\verb|answer|: \_\_\_\_\_\_\_\_\_\_\_\_\_\_\_\_\_\_\_\_\_\_\_\_\_\_\_\_\_\_\_\_

\end{enumerate}


\section{Word Sense Disambiguation}

Consider the sentence

\begin{minted}[bgcolor=LightGray,fontsize=\footnotesize]{text}
The quick brown fox had been already domesticated and become like the common dog:
unpleasant and prehistoric like the lazy wolf.
\end{minted}

And the definitions of dog (from the Wordnet):

\begin{minted}[bgcolor=LightGray,fontsize=\footnotesize]{text}
dog.n.01: a member of the genus Canis (probably descended from the common wolf) that
          has been domesticated by man since prehistoric times; occurs in many breeds

dog.n.02: a dull unattractive unpleasant girl or woman
\end{minted}

Run the Lesk algorithm, showing the results for each of the definitions.

\verb|Count for dog.n.01|: \_\_\_\_\_\_\_\_\_\_\_\_\_\_\_\_\_\_\_\_\_\_\_\_\_\_\_\_\_\_\_\_

\verb|Count for dog.n.02|: \_\_\_\_\_\_\_\_\_\_\_\_\_\_\_\_\_\_\_\_\_\_\_\_\_\_\_\_\_\_\_\_

\verb|Final answer|: \_\_\_\_\_\_\_\_\_\_\_\_\_\_\_\_\_\_\_\_\_\_\_\_\_\_\_\_\_\_\_\_


\end{document}

